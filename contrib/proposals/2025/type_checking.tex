\documentclass{article}

\usepackage[utf8]{inputenc}
\usepackage[english]{babel}
\usepackage{hyperref}
\usepackage{amsmath}
\usepackage{float}
%\usepackage{cite}
\usepackage{url}
\usepackage{listings}
\usepackage{graphicx}
\usepackage{color}
\usepackage{tabularx}
\usepackage{booktabs}
\usepackage{titlesec}
\usepackage[a4paper, top=2.5cm, bottom=2.5cm, left=2.5cm, right=2.5cm]{geometry}
\usepackage{titling}

\hypersetup{
    hidelinks,
    colorlinks,
    urlcolor=blue,
    citecolor=[rgb]{0,.6,.25}
}

\definecolor{codegreen}{rgb}{0,0.6,0}
\definecolor{codegray}{rgb}{0.5,0.5,0.5}
\definecolor{codepurple}{rgb}{0.58,0,0.82}
\definecolor{backcolour}{rgb}{0.95,0.95,0.92}
\definecolor{lightgray}{rgb}{.9,.9,.9}
\definecolor{darkgray}{rgb}{.4,.4,.4}
\definecolor{purple}{rgb}{0.65, 0.12, 0.82}

\title{
    BTI2029 - Project submission for Open Source contribution\\
    \large
    ~\\
    \textbf{Verify and improve correctness of Vula with Python type annotations}
}
% Manually set in title.tex
\author{}
\date{}
% Left align title
\pretitle{\begin{flushleft}\LARGE}
\posttitle{\end{flushleft}}
\preauthor{\begin{flushleft}\large}
\postauthor{\end{flushleft}}
\predate{\begin{flushleft}}
\postdate{\end{flushleft}}
% Adjust this value to move the title up
\setlength{\droptitle}{-2cm}
% This removes page numbers
\pagenumbering{gobble}

\begin{document}
    \maketitle
    \vspace{-2cm}

    \begin{table}[H]
        \begin{tabularx}{1\textwidth}{l X}
            \toprule
            School & \textbf{Bern University of Applied Sciences} \\
            \midrule
            Module & \textbf{BTI2029 - Open Source Software Contribution} \\
            \midrule
            Instructor & \textbf{Dr. Jacob Appelbaum} \\
            \midrule
            Authors  & \textbf{Gauss Frank (gausf1), Severin Lötscher (lotss3), Lukas von Allmen (vonal3)} \\
            \midrule
            Date  & \textbf{\today} \\
            \bottomrule
        \end{tabularx}
        \label{tab:general-information}
    \end{table}
    % What do we want to do?
    \section{Objectives}\label{sec:objectives}

    Verify and improve the correctness of the existing python code base in \href{https://vula.link/}{vula} by:
    \begin{itemize}
        \item using type checkers to identify missing or wrong type annotations
        \item adding and improving type annotations
        \item enforcing type checking in different stages of the development process
    \end{itemize}

    % Why we want to do this?
    \section{Reasoning}\label{sec:reasoning}
    Type annotations improve a code base in readability, stability, maintainability, security and correctness.

    % How we are doing it?
    \section{Implementation}\label{sec:implementation}
    We will use \href{https://www.mypy-lang.org/}{mypy} to identify missing or wrong type annotations in the current code base.
    \hfill \break
    \hfill \break
    Further we will introduce an additional type checker for cross verification: \\
    \href{https://github.com/microsoft/pyright/blob/main/docs/mypy-comparison.md}{pyright - Differences Between Pyright and Mypy} \\
    \hfill \break
    Based on the inputs of our instructor and maintainer of vula we will adopt the following best practices:
    \begin{itemize}
        \item Instead of creating a custom mypy configuration, we will use the `--strict` flag.
        \item Our scope is to type the existing code base, changing code will be our last resort.
        \item For untyped imports, we check if there are stubs available,
        otherwise we will ignore it as described in: \href{https://mypy.readthedocs.io/en/stable/running_mypy.html#missing-imports}{mypy - Missing imports}
        \item Finding the right middle ground regarding type specificity: `Any` or `object` is almost never the right choice, but `Iterable[str]` might be a better description than `list[str]` for a functions argument.
    \end{itemize}

    \section{Goals}\label{sec:goals}
    \paragraph{\textbf{Minimum Goals:}}
    \begin{itemize}
        \item Resolve all issues raised by mypy --strict.
        \item Add a second type checker for cross verification: \href{https://github.com/microsoft/pyright/blob/main/docs/mypy-comparison.md}{pyright}
        \item Add the type checkers to the CI configurations by extend the existing command `make check`.
        \item Add a git pre-commit hook which runs the type checkers on each commit (e.g.: \href{https://pre-commit.com/}{pre-commit}).
    \end{itemize}

    \paragraph{\textbf{Optional goals:}}
    \begin{itemize}
        \item Add python type stubs for dependencies (Ad-hoc type stubs in the vula codebase).
        \item Add type annotations for dependencies (Contribution to a third party library).
    \end{itemize}
\end{document}