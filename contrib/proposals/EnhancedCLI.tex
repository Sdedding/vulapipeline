\documentclass[12pt]{article}

% Packages
\usepackage{graphicx}
\usepackage{hyperref}
\usepackage{geometry}
\geometry{a4paper, margin=1in}

\title{Vula Project Proposal: Enhancing the CLI for System Administrators}
\author{Vaney Quentin Guillaume, Noah Ferrari, Joel Meier}
\date{\today}

\begin{document}
	
	\maketitle
	
	\textbf{Overview:} The current Vula CLI lacks features vital to sysadmins. We propose a redesign focused on scriptability, accessibility, and automation. Key improvements include structured output (YAML/JSON), shell autocomplete, colorblind-friendly themes, and a new `vula script` subcommand tailored for automation and Ansible integration.
	
	\textbf{User Story:} Frank Hank, a sysadmin, manages multiple Vula instances and needs reliable, automatable tools. He requires structured output, accessible CLI design, autocomplete, and clear scripting integration to reduce manual overhead.
	
	\textbf{Objectives (Success Criteria):}
	\begin{enumerate}
		\item Redesign Vula CLI for scriptability.
		\item Add YAML/JSON output formats.
		\item Offer selectable colorblind-safe themes.
		\item Implement shell autocomplete (e.g., Click).
		\item Include full documentation with doctests and CLI design.
	\end{enumerate}
	
	\textbf{Key Feature:} The `vula script` subcommand will act as a stable automation interface with goal-oriented commands and YAML output, enabling use in Ansible and scripts without affecting other CLI parts.
	
	\textbf{Security and Accessibility:} We treat accessibility as security. Improvements include:
	\begin{itemize}
		\item Non-color indicators (symbols/patterns).
		\item Multiple color schemes for common vision deficiencies.
		\item Structured, machine-readable output.
	\end{itemize}
	
	\textbf{Extended Goals:}
	\begin{itemize}
		\item Document a Vula deployment with Ansible.
		\item Provide a cron job example using `vula script`.
	\end{itemize}
	
\end{document}

