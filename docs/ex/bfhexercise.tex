%============================ MAIN DOCU =================================
% define document class
\PassOptionsToPackage{table}{xcolor}
\documentclass[
	invert-title=true,
	titlepage=false,
	titleimage-ratio=13
]{bfhpub}				% KOMA-script report
%---------------------------------------------------------------------------
% Documents paths
%---------------------------------------------------------------------------
\makeatletter
\def\input@path{{content/}}
%or: \def\input@path{{/path/to/folder/}{/path/to/other/folder/}}
\makeatother
%-----------------  Base packages     --------------------------------------
% Include Packages
\usepackage[french,ngerman,main=english]{babel}

%% To disable the french list setting you can add -- see https://gitlab.ti.bfh.ch/bfh-latex/bfh-ci/-/issues/166
%\frenchsetup{StandardLists=true}
%---------------------------------------------------------------------------
\usepackage{geometry}
\geometry{
   a4paper, % Papierformat
   top=3cm, bottom=4cm, % Rand oben/unten
   outer=2.5cm, inner=3cm % aussen/inne
}
%---------------------------------------------------------------------------
% Hyperref Package (Create links in a pdf)
%---------------------------------------------------------------------------
\usepackage[
   hidelinks
  ,colorlinks
  ,linkcolor=.
  ,filecolor=BFH-MediumGreen
  ,urlcolor=BFH-MediumBlue
  ,plainpages=false
  ,pdfpagelabels
  ,pdfusetitle
  ,hypertexnames = {true},	% no failures "same page(i)"
]{hyperref}

\LoadBFHModule{boxes}

\usepackage{multicol}

\newcommand*{\key}[1]{\enquote{\texttt{#1}}}
\newcommand*{\code}[1]{\enquote{\texttt{#1}}}

%---------------------------------------------------------------------------
\ihead*{\csname@title\endcsname{}}
\chead*{}
\ohead*{\csname@subtitle\endcsname{}}

\setcounter{secnumdepth}{0}
\setlength{\parskip}{0pt}
\setlength{\parindent}{0pt}

\colorlet{BFH-Title}{white}

\begin{document}
%----------------  BFH tile page   -----------------------------------------
\title{The Title}
\subtitle{Some Subtitle}
\author{The Author}
\subject{Subject}
%  \publishers{publishers}
\institution{Bern University of Applied Sciences}
\department{Technik und Informatik}
\institute{Mikro- und Medizintechnik}
\version{0.1}
%The starred variant will automatically scale and clip the image. the non-starred one will allow you to set the size yourself
%  \titlegraphic*{\includegraphics{example-image}}
  
\maketitle
%------------ TABLEOFCONTENTS ---------------
%  \tableofcontents

\section*{Abstract}
Some meaningful abstract.

\section*{Objectives}
\begin{multicols}{2}
\begin{itemize}
\item ...
\item ...
\item ...
\item ...
\end{itemize}
\end{multicols}

\section*{Description}
The exercise description....

Maybe something you would like to highlight.

\begin{bfhNoteBox}
  A note box.
\end{bfhNoteBox}

\begin{bfhRecycleBox}
  A recycle box.
\end{bfhRecycleBox}

\begin{bfhReadBox}
  A read box.
\end{bfhReadBox}

\section*{Outcomes}
\begin{multicols}{2}
\begin{itemize}
\item ...
\item ...
\item ...
\item ...
\end{itemize}
\end{multicols}


\end{document}
