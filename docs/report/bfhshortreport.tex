%============================ MAIN DOCU =================================
% define document class
\PassOptionsToPackage{table}{xcolor}
\documentclass[
   fontsize=10.5pt,
   invert-title=true,
   titlepage=false,
   titleimage-ratio=13,
   class=article,
   twocolumn,
   parskip=half-
]{bfhpub}				% KOMA-script report
%---------------------------------------------------------------------------
% Documents paths
%---------------------------------------------------------------------------
\makeatletter
\def\input@path{{content/}}
%or: \def\input@path{{/path/to/folder/}{/path/to/other/folder/}}
\makeatother
%-----------------  Set default font family  -------------------------------
\renewcommand{\familydefault}{\sfdefault}
%-----------------  Base packages     --------------------------------------
% Include Packages
\usepackage[french,ngerman,main=english]{babel}
%% To disable the french list setting you can add -- see https://gitlab.ti.bfh.ch/bfh-latex/bfh-ci/-/issues/166
%\frenchsetup{StandardLists=true}
\usepackage[babel, german=quotes]{csquotes}
%---------------------------------------------------------------------------
\usepackage{blindtext}  %% For placeholder text only
%---------------------------------------------------------------------------
\usepackage{geometry}
\geometry{
   a4paper, % Papierformat
   top=3cm, bottom=4cm, % Rand oben/unten
   outer=2.5cm, inner=3cm % aussen/inne
}
%---------------------------------------------------------------------------
\usepackage[
  backend=biber,
  style=alphabetic,
  sorting=ynt
]{biblatex}
\addbibresource{sample.bib} %% Name of bibliography file
%---------------------------------------------------------------------------
% Hyperref Package (Create links in a pdf)
%---------------------------------------------------------------------------
\usepackage[
   hidelinks
  ,colorlinks
  ,linkcolor=.
  ,filecolor=.BFH-MediumGreen
  ,urlcolor=BFH-MediumBlue
  ,citecolor=.
  ,plainpages=false
  ,pdfpagelabels
  ,pdfusetitle
  ,hypertexnames = {true},	% no failures "same page(i)"
]{hyperref}
%---------------------------------------------------------------------------

\newcommand*{\code}[1]{\enquote{\texttt{#1}}}  %% The "code" macro definition

\ohead*{\csname@author\endcsname{}}
\chead*{}
\ihead*{\csname@title\endcsname{}\title}

\setcounter{secnumdepth}{0}

\LoadBFHModule{boxes,terminal}

\begin{document}
%----------------  BFH tile page   -----------------------------------------
  \title{Title}
  \subtitle{Sub Title}
  \author{Me, myself and I}
  \subject{BTF3232}
  %\publishers{publishers}
  \institution{Bern University of Applied Sciences}
  \department{Technik und Informatik}
  \institute{Mikro- und Medizintechnik}
  \version{1.0.0}
  %The starred variant will automatically scale and clip the image. the non-starred one will allow you to set the size yourself
  

\makeatletter
\twocolumn[
   \begin{@twocolumnfalse} 
     \maketitle 
     %\section{Abstract}
     \textbf{\blindtext[1]}
    \end{@twocolumnfalse}
    \vskip1.5em
]
\makeatother


%\section*{Abstract}
%\textbf{\blindtext[1]}


%\begin{multicols}{2}
\section{Introduction}
\blindtext[1]

\section{Methodology}

\setupLinuxPrompt{student}
\begin{ubuntu}
echo hello `\StartConsole`
hello
\end{ubuntu}

Using \texttt{biblatex} you can display bibliography divided into sections, depending of citation type. 
Let's cite! The Einstein's journal paper \cite{einstein} and the Dirac's book \cite{dirac} are physics related items. 
Next, \textit{The \LaTeX\ Companion} book \cite{latexcompanion}, the Donald Knuth's website \cite{knuthwebsite}, \textit{The Comprehensive Tex Archive Network} (CTAN) \cite{ctan} are \LaTeX\ related items; but the others Donald Knuth's items \cite{knuth-fa,knuth-acp} are dedicated to programming.

\begin{table}
\centering
%\colorlet{BFH-tablehead}{BFH-MediumBlue!50}
%\colorlet{BFH-table}{BFH-MediumBlue!15}
\begin{bfhTabular}{lll}
   Header 1 &Header 2 &Header 3 \\
   \hline
   Content 11&Content 12 & Content 13 \\
   Content 21&Content 22 & Content 23
\end{bfhTabular}
\end{table}


This is a text with an inline CLI command \code{gcc -o hello -Wall hello.c} and some inline C code \code{\#define MY\_CONST 1234}.
\section{Results}

\blindtext[1]
\begin{bfhAlertBox}
  An alert box.
\end{bfhAlertBox}
\blindtext[1]

\begin{bfhWarnBox}
  A warning box.
\end{bfhWarnBox}
\blindtext[1]

\begin{bfhNoteBox}
  A note box.
\end{bfhNoteBox}

\section{Discussion and Conclusion}
\blindtext[1]

%\end{multicols}

%------------ BIBLIOGRAPHY ---------------

\onecolumn
\printbibliography



\end{document}
