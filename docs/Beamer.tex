% -*- mode: latex; -*-
% Vula Mock-GUI – Quick Local Test Setup
% Using BFH Beamer corporate-design class
% ------------------------------------------------------------
\documentclass[
  ngerman,          % BFH class language option (colors, fonts)
  authorontitle=true,
]{bfhbeamer}

% Slides are written in English, but we leave the ngerman option for the CI theme
\usepackage[english]{babel}

\usepackage{iftex}
\ifPDFTeX
  \usepackage[utf8]{inputenc}
\fi

% Code blocks
\usepackage{listings}
\lstset{
  basicstyle=\ttfamily\small,
  frame=single,
  breaklines=true,
  columns=fullflexible
}

% ------------------------------------------------------------
% Meta data
% ------------------------------------------------------------
\title{Vula Mock-GUI: Rapid Front-End Prototyping}
\subtitle{Run the GUI without DBus or kernel dependencies}
\author{Mursel Khan, Sander Dedding}
\institute{Vula Project}

% ------------------------------------------------------------
\begin{document}

\maketitle

% ------------------------------------------------------------
\section{Why a Mock-GUI?}

\begin{frame}{Motivation}
  \begin{itemize}
    \item \textbf{Isolate} the front end from system services (DBus, WireGuard, \dots).
    \item Enable \textbf{UI development}, \textbf{bug reproduction} and \textbf{CI tests} on any machine.
    \item Provide deterministic dummy data via a \texttt{MockDataProvider}.
  \end{itemize}
\end{frame}

% ------------------------------------------------------------
\section{Architecture}

\begin{frame}{Key Components}
  \begin{description}
    \item[MockDataProvider] Returns empty or sample structures for \texttt{get\_peers()}, \texttt{get\_prefs()}, \dots
    \item[Patching] Functions \texttt{patch\_dataprovider()} and \texttt{patch\_constants()} monkey-patch the real modules before the UI loads.
    \item[Tk App] After patching, \texttt{vula.frontend.ui.App} starts as usual—no code changes inside the UI layer.
  \end{description}
\end{frame}

% ------------------------------------------------------------
\section{Usage inside Dev-Container}

% Frame with code block – needs [fragile]
\begin{frame}[fragile]{One-Time Container Setup}
  \framesubtitle{Executed automatically via \texttt{postCreateCommand}}
\begin{lstlisting}[language=bash]
# Install GTK and other build dependencies
sudo apt-get update && sudo apt-get install -y gir1.2-gtk-3.0 \
  girepository-2.0 libgirepository1.0-dev libcairo2-dev libffi-dev pkg-config

# Create and activate virtual environment
python3 -m venv venv
source venv/bin/activate

# Install Vula in editable mode and Python deps
pip install --upgrade pip setuptools wheel
pip install -e .

pip install pydbus PyGObject PyYAML qrcode Pillow pynacl cryptography schema click
\end{lstlisting}
\end{frame}

% Second frame with enumerate and code – needs [fragile]
\begin{frame}[fragile]{Start the Mock-GUI}
\begin{enumerate}
  \item Open browser: \url{http://localhost:6080} (noVNC desktop)
  \item In terminal:
\begin{lstlisting}[language=bash]
cd /workspaces/vula
source venv/bin/activate
python tools/mock_gui_main.py
\end{lstlisting}
  \item The GUI window appears; interact freely—backend is mocked.
\end{enumerate}
\end{frame}

% ------------------------------------------------------------
\section{Typical Workflows}

\begin{frame}{Developer Benefits}
  \begin{itemize}
    \item \textbf{Styling tweaks}: edit \texttt{vula/frontend/ui/*.py}, reload.
    \item \textbf{Debugging}: set breakpoints without touching system services.
    \item \textbf{CI}: headless launch, capture screenshots, run UI tests.
  \end{itemize}
\end{frame}

% ------------------------------------------------------------
\section{Summary}

\begin{frame}{Take-aways}
  \begin{itemize}
    \item Mock-GUI \emph{decouples} front-end work from privileged backend.
    \item Perfect for fast iteration, demonstrations, and automated tests.
    \item Easily extended—add more fake data or test scenarios as needed.
  \end{itemize}
\end{frame}

\end{document}
